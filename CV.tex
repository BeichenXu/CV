%%%%%%%%%%%%%%%%%%%%%%%%%%%%%%%%%%%%%%%%%
% "ModernCV" CV and Cover Letter
% LaTeX Template
% Version 1.3 (29/10/16)
%
% This template has been downloaded from:
% http://www.LaTeXTemplates.com
%
% Original author:
% Xavier Danaux (xdanaux@gmail.com) with modifications by:
% Vel (vel@latextemplates.com)
%
% License:
% CC BY-NC-SA 3.0 (http://creativecommons.org/licenses/by-nc-sa/3.0/)
%
% Important note:
% This template requires the moderncv.cls and .sty files to be in the same 
% directory as this .tex file. These files provide the resume style and themes 
% used for structuring the document.
%
%%%%%%%%%%%%%%%%%%%%%%%%%%%%%%%%%%%%%%%%%

%----------------------------------------------------------------------------------------
% PACKAGES AND OTHER DOCUMENT CONFIGURATIONS
%----------------------------------------------------------------------------------------

\documentclass[11pt,a4paper,sans]{moderncv} % Font sizes: 10, 11, or 12; paper sizes: a4paper, letterpaper, a5paper, legalpaper, executivepaper or landscape; font families: sans or roman

\moderncvstyle{classic} % CV theme - options include: 'casual' (default), 'classic', 'oldstyle' and 'banking'
\moderncvcolor{blue} % CV color - options include: 'blue' (default), 'orange', 'green', 'red', 'purple', 'grey' and 'black'

\usepackage[scale=0.85]{geometry} % Reduce document margins
\linespread{0.9}

%----------------------------------------------------------------------------------------
% NAME AND CONTACT INFORMATION SECTION
%----------------------------------------------------------------------------------------

\firstname{Beichen} % Your first name
\familyname{Xu} % Your last name

% All information in this block is optional, comment out any lines you don't need
\title{Curriculum Vitae}
\address{44 Princes Gate}{London, UK, SW7 2QA}
\mobile{+44 0741 091 5752}
\email{xb223@ic.ac.uk}

%----------------------------------------------------------------------------------------

\begin{document}
\makecvtitle % Print the CV title

%----------------------------------------------------------------------------------------
% EDUCATION SECTION
%----------------------------------------------------------------------------------------

\section{Education}

\cventry{2023-2024}{MRes Machine Learning and Big Data in the Physical Sciences}{Imperial College London}{Department of Physics}{}{}
\cventry{2020--2023}{B.Sc. Physics}{University of Birmingham}{School of Physics and Astronomy}{}{}

%----------------------------------------------------------------------------------------
% WORK EXPERIENCE SECTION
%----------------------------------------------------------------------------------------

\section{Work Experience}

\subsection{Internships}

\cventry{2022}{Summer Intern}{\textsc{Kavli Institute of Cosmology Cambridge}}{Cambridge}{}{
\begin{itemize}
    \item Investigated machine learning-enhanced Bayesian inference.
    \item Used nested sampling to train cosmology data and masked aggressive flow to generate the posterior of different parameters.
    \item Developed and trained machine learning models to enhance data analysis accuracy.
    \item Created visualizations, including 2D plots and GIFs, to illustrate AI training processes.
    \item Contributed to the "margarine" Python package on GitHub by adding a new objective function.
    \item Gained proficiency in computing tools such as Linux, vim, vimteractive, ssh, tmux, tqdm, and TensorFlow, and studied Python packages like margarine and anesthetic.
\end{itemize}}

%------------------------------------------------

\cventry{2021}{Summer Intern}{\textsc{Purple Mountain Observatory}}{Nanjing}{}{
\begin{itemize}
    \item Analyzed Gamma Ray Burst (GRB) data to investigate their causes and mechanisms.
    \item Utilized Python for data fitting and analysis, focusing on Swift GRB data.
    \item Enhanced programming skills in Python and the iminuit library.
\end{itemize}}

%----------------------------------------------------------------------------------------
% PROJECT EXPERIENCE SECTION
%----------------------------------------------------------------------------------------

\section{Project Experience}

\subsection{Academic}

\cventry{2024}{MRes Project}{\textsc{Imperial College London}}{London}{}{
\begin{itemize}
    \item Created simulated multi-source gravitational wave time-series data.
    \item Developed and trained WGAN models to analyze gravitational wave signals.
    \item Utilized Imperial College London's HPC for training and performed data visualization.
    \item Debugged neural networks to improve performance and accuracy.
    \item Employed PyTorch, PyTorch Lightning, and GetDist for model development, training and analysis.
\end{itemize}}

%----------------------------------------------------------------------------------------
% AWARDS SECTION
%----------------------------------------------------------------------------------------

\section{Awards}

\cvitem{2018}{18th Award Program for Future Scientists -- Second Prize}
\cvitem{2019}{Physics Olympiad in Jiangsu Province, China -- First prize}

%----------------------------------------------------------------------------------------
% COMPUTER SKILLS SECTION
%----------------------------------------------------------------------------------------

\section{Computer Skills}

\cvitem{Programming}{Python, C++, Mathematica, PyTorch, TensorFlow, SQL, Lightning}
\cvitem{Computing}{Unix, Bash, vim, git, \LaTeX, FPGA (VHDL/Verilog)}
\cvitem{OS}{Linux, Windows}

%----------------------------------------------------------------------------------------

%----------------------------------------------------------------------------------------
% LANGUAGES SECTION
%----------------------------------------------------------------------------------------

\section{Languages}

\cvitemwithcomment{Chinese}{Native}{}
\cvitemwithcomment{English}{Intermediate}{Conversational fluency}

%----------------------------------------------------------------------------------------

\section{Publications}
\cvitem{1}{Xu, Beichen, Jun Su, and Weiguo Wang. "An expanding balloon: a small universe." \textit{Physics Education} 53.6 (2018): 065005.}

\end{document}
